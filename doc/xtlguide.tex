% XTL - the eXternalization Template Library
% Copyright (C) 1998, 1999 Jose' Orlando Pereira, Universidade do Minho
%
% jop@di.uminho.pt - http://gsd.di.uminho.pt/~jop
%
% Departamento de Informatica, Universidade do Minho
% Campus de Gualtar, 4710-057 Braga, Portugal
%
% This library is free software; you can redistribute it and/or
% modify it under the terms of the GNU Library General Public
% License as published by the Free Software Foundation; either
% version 2 of the License, or (at your option) any later version.
%
% This library is distributed in the hope that it will be useful,
% but WITHOUT ANY WARRANTY; without even the implied warranty of
% MERCHANTABILITY or FITNESS FOR A PARTICULAR PURPOSE.  See the GNU
% Library General Public License for more details.
%
% You should have received a copy of the GNU Library General Public
% License along with this library; if not, write to the Free
% Software Foundation, Inc., 59 Temple Place - Suite 330, Boston,
% MA 02111-1307, USA
%
% $Id: xtlguide.tex,v 1.3 2001/07/05 09:48:30 jop Exp $

\documentclass[titlepage,twocolumn]{article}
\usepackage{isolatin1}
\usepackage{epsfig}
\usepackage{fullpage}
\usepackage{pslatex}

\title{XTL -- The Externalization Template Library}
\author{Jos� Orlando Pereira\\
{\small Departamento de Inform�tica}\\
{\small Universidade do Minho}\\[1cm]
}

\date{%
{\small \textit{Address: Dep. de Inform�tica, Campus de Gualtar, 4710-057 Braga, PORTUGAL}}\\
{\small \textit{Phone: +351 253 604 477 / Fax: +351 253 604 471}}\\
{\small \textit{E-mail: jop@di.uminho.pt}}\\[1cm]
December 1999\\[1cm]
}

\newcommand{\xtl}{XTL}

%\renewcommand{\baselinestretch}{1.62} %12pt
\renewcommand{\baselinestretch}{1.22} %12pt % Uma batotazita...

\begin{document}

\maketitle

\begin{abstract}
  The Externalization Template Library (\xtl) is a set of template
  classes designed to ease the task of converting C++ data-structures
  to a language and machine independent external representation, and
  from this external representation back to C++ data-structures, as
  required for communication or persistent storage in heterogeneous
  computer systems.
  
  As externalization requires the traversal of data-structures, most
  existing solutions require a special compiler to generate traversal
  procedures from simple specifications, restricting which
  data-structures can be externalized. Those that do not, are either
  inefficient or require extra effort by the programmer.
  
  In contrast, the \xtl\ demonstrates how some features of modern C++
  can make the development of traversal procedures for externalization
  convenient without restricting the programmer to a few
  data-structures. As the \xtl\ is especially designed for optimizing
  compilers, the resulting code is also very fast.
\end{abstract}

\section{Introduction}

The tasks of converting data-structures to a language and machine
independent external representation and from this external
representation back to the original data-structures are respectively
known as \emph{externalization} and
\emph{internalization}\footnote{Externalization is also known as
  marshaling, serialization, linearization, or pickling.}.

The external representation of a data-structure is a sequence of bytes
which exhibits an important characteristic: given the appropriate
internalization procedure, it is sufficient to rebuild the
data-structure regardless of where it was created. As such,
standardized external data formats and the associated externalization
and internalization procedures are an absolute requirement for
communication or persistent storage in heterogeneous computer systems.

Most of the existing toolkits for externalization require a
data-description language and its associated compiler to generate fast
conversion procedures from simple specifications of data-structures.
This is the case of most distributed programming systems which rely on
interface description languages.

However, this solution poses several problems, such as tying
externalization to a specific function in a particular distributed
computing toolkit and making it unusable for any other purpose. In
addition, the mapping from the specification to concrete
data-structures is usually also very restrictive in which
data-structures of the programming language are supported. In the
context of object-oriented programming, it is also a problem that the
internal structure of objects has to be exposed.

On the other hand, solutions that do not rely on a special compiler,
are either inefficient or require the programmer to perform the
tedious task of hand-writing and optimizing externalization and
internalization procedures for each data-structure. The lack of a
language independent specification is also a severe drawback when
interoperability between different implementations is desired.

Nonetheless, by using the target programming language directly, the
freedom of the programmer to choose which data-structure is fit for
each purpose, regardless of their external format, is preserved, in
contrast to being imposed by the data-description compiler and it can
be hidden as necessary. In some demanding applications, this added
flexibility can be more desirable than a language independent
specification, especially if the external format is standardized and
well known.

In this context, the Externalization Template Library (\xtl),
demonstrates how a careful use of modern C++ features
\cite{stroustrup,isocpp} can be used to make the development of
efficient conversion procedures directly in the programming language
convenient to the programmer without compromising on performance.  It
is also shown how the resulting traversal procedures can also be used
as generic traversals for several other purposes.

The rest of this text is structured as follows: Section
\ref{sec:design} outlines some design issues and tradeoffs considered
in the design of the \xtl; Section \ref{sec:arch} presents an overview
of the architecture; Section \ref{sec:api} depicts the application
programmer's interface to the \xtl, showing how it is comparable in
simplicity to a data-description language, while being significantly
more powerful and flexible as is shown in Section \ref{sec:adv};
Section \ref{sec:perf} examines the performance of the \xtl, including
a benchmark and its evaluation; Section \ref{sec:extre} describes some
applications of the \xtl\ and discusses the possibility of using
procedures written with it as a generic traversal mechanism for C++
data-structures.

\section{Design issues}
\label{sec:design}

The design of general purpose externalization systems poses two kinds
of problems. The first is how to represent each of the data elements
externally, so that they can be correctly decoded when being
internalized. This problem is not addressed here, as the
\xtl\ can be configured to generate and decode several widely accepted
external formats.

The second problem is how to traverse data-structures in order to
externalize each element contained in them and, when internalizing,
how to correctly rebuild complex data-structures from the external
``flat'' representation.

There are two possible solutions to this problem: the developer either
explicitly writes procedures to do the traversal or uses a data
description compiler to generate both the data-structure definition
and the associated traversal procedures from a simple
specification\footnote{In fact, there is a third solution which is
  valid only for languages with reflection \cite{kiczales}, where the
  traversal procedures can be programmed at the meta-level as happens
  with Java serialization \cite{pickle}. However, as this is not valid
  for C++ it is not further discussed here.}.

Using a data description language to describe data-structures is
usually easier for the programmer and the resulting externalization
code, being machine generated, can sacrifice readability and be
optimized for efficiency.

However, the usage of a data description language has some drawbacks,
especially in the context of object-oriented programming and for
persistence. The first of these is that usually they are tightly
coupled to a particular distributed programming toolkit and are not
usable for other purposes, as for instance, persistent storage.

The second problem is that data-structure definitions must be generated
by the special compiler and as such, the programmer looses the freedom
to hide them inside class definitions. In an object-oriented
environment this is not desired but the alternative is to use an
intermediate data-structure which is not encapsulated and then convert
it back and forth to the encapsulated structure. This extra conversion
step is both awkward and inefficient.

Because data description languages aim to be independent from specific
programming languages, there is also the problem of mapping high-level
specifications in the data description language to low-level
data-structures in a programming language. For instance, a sequence in
the CORBA interface definition language is usually mapped to an array
\cite{corba}.  However, this seldom is satisfactory for the code that
processes it, which might require something different, for example, an
Standard Template Library (STL) container \cite{stl}.  Note that
selecting the STL container, or any other data-structure, for that
matter, wouldn't solve the problem, as there are situations where the
array is precisely what is required.

A possible solution to this problem is to complement the data
description with programming language specific configuration.  The
configuration language conveys further information about the mapping
from abstract specifications to concrete constructs in a specific
programming language.  Although this would be an improvement over a
simple data description language, it would introduce yet another
degree of complexity and somewhat defeat the purpose of language
independence in data-descriptions. It can also be pointed out, that in
the extreme, this configuration language would have to be as powerful
as the target programming language, to allow for every possible
mapping.

Hand-written externalization procedures do not suffer from any of
these limitations. On the other hand, they have some of their own. The
most notable of these is the dependence on a specific programming
language, which is especially important when externalization is used
for heterogeneous distributed programming. Nonetheless, this
limitation can be somewhat circumvented by using standard external
formats.

In addition, writing carefully hand-optimized code to produce fast
externalization and internalization procedures, is usually a complex,
error-prone and time consuming task, especially if separate
externalization and internalization procedures must be written for
each data-structure, as happens, for instance, with the CORBA
Externalization Service \cite{corba-es} and with almost all others.

On the other hand, if some framework that enables the programmer to
write simple externalization procedures is available, it may severely
impact the possibility for compiler type-checking and optimization.
This is the case of the original XDR library by Sun \cite{sunxdr},
which has pre-defined procedures for the most common composite
data-structures and where the same procedure can be used both for
externalization and internalization.\footnote{Most interestingly, the
  associated compiler \texttt{rpcgen}, outputs special inline code to
  improve performance. As a result, the generated code looses the
  simplicity and elegance of its less efficient counterpart.}.

In short, the conflicting design goals are ease of use, performance,
flexibility and interoperability. By using a special compiler, it is
possible to achieve most of them, except flexibility as it restricts
which data-structures can be used, what is the purpose of
externalization and poses several problems in the context of
object-oriented programming.

Hand-written externalization procedures provide maximum flexibility.
However, easy to use frameworks tend to be inefficient and
interoperability is dependent on the standardization of external
representations.

In this context, the \xtl\ provides a way to
write externalization procedures directly in the C++ programming
language, which are very simple to write but efficient. It also
supports most features of C++, including some widely used programming
idioms and several standard external data-formats.

\begin{figure}[t]
  \begin{center}
    \leavevmode
    \epsfig{file=arch.ps}
    \caption{Overview of the \xtl\ architecture.}
    \label{fig:arch}
  \end{center}
\end{figure}

\section{Architecture}
\label{sec:arch}

The \xtl\ is composed of three layers. The bottommost layer is the
\emph{buffer driver} which is responsible for storing the external
representation produced by the middle layer\footnote{And for
  retrieving the external representation and feeding it to the upper
  layer, when internalizing. For the sake of simplicity, we describe
  the architecture solely in terms of externalization. The
  internalization process is symmetrical and can easily be inferred.}.
Currently, there are three options for this layer, which store data in
memory buffers, C-style file streams or a C++-style streams.

The middle layer is the \emph{format driver} and defines the external
data format. It accepts requests to externalize each of the basic data
types (e.g. numbers) and provides some primitives to allow the
description of complex data-structures. Note that these complex data
structures are abstract data-structures, such as fixed or varying
length sequences. The available formats are: CORBA Generic Inter-ORB
Protocol Common Data Representation (GIOP CDR) \cite{corba}, CORBA
Externalization Service \cite{corba-es}, RFC1832 External Data
Representation (XDR) \cite{rfcxdr} and plain text

The mapping of these high-level concepts to concrete data-structures in
the C++ language is done at the topmost layer, the \emph{object
  stream}. For instance, it is possible to map sequences to arrays or
containers. 

The object stream layer is also the visible programming interface to
the \xtl\ and defines how ``easy'' is to write externalization
procedures.  In this context, ``easy'' means several different things:
\begin{itemize}
\item how much code the programmer has to write;
\item how much the programmer has to learn to write the code;
\item how much help the programmer gets from the compiler to ensure
  the correctness and efficiency of the code.
\end{itemize}
The interface of the \xtl\ is designed to target all these. First, as
the programming interface is somewhat modeled upon the original XDR
library, a single procedure for each data type is enough for both
externalization and internalization. This happens because
externalization procedures are in fact general purpose traversals of
the data-structure, which can be used for other purposes. For
instance, the XDR library also uses them to free memory.

The resemblance to XDR will also help many programmers accustomed to
it to quickly understand how to write externalization procedures with
the \xtl. However, the \xtl\ takes advantage of function overloading
to make it even simpler than XDR, as all basic C++ data types and most
of the composite data types are externalized with the same method
name. Different method names are only necessary when there is the need
to disambiguate a C++ construct that has several distinct meanings.
For instance, \texttt{char*} can be a reference to a single character;
an optional character, absent if \texttt{NULL}; a zero-terminated
string or even raw data.

The use of template functions, instead of generic pointers and
function parameters, allows the compiler to further check the
correctness of the code, ensuring that each data-structure is accessed
only by the correct externalization procedures\footnote{Templates are
  also essential for efficiency. See Section \ref{sec:perf} for further
  discussion of this topic.}.

Finally, the \xtl\ directly supports most of the features of C++, such
as template classes; externalization through pointers to, possibly
abstract, base classes, STL containers and container-style strings. In
addition, as externalization procedures are methods of the respective
classes, no internal structure needs to be shown and encapsulation is
preserved\footnote{There is however the possibility to write the
  externalization procedures as global functions, which is useful for
  externalizing objects of classes which must not be modified. See
  Section \ref{sec:advspec} for an example.}.

\section{Programmer's interface}
\label{sec:api}

\subsection{Streams}

The externalization process is initiated by creating a \emph{stream}.
This is done by selecting the appropriate buffer and format drivers
and by stacking them together. For example, a CORBA GIOP CDR output
stream to memory is declared as:
{\small\begin{verbatim}
    char buf[SIZE];
    mem_buffer mb(buf, SIZE);
    GIOP_format<mem_buffer> gfmb(mb);
    obj_output<GIOP_format<mem_buffer> >
        stream(gfmb);
\end{verbatim}}
An XDR input stream from a file is declared as:
{\small\begin{verbatim}
    FILE* fp=fopen("stuff", "r");
    cfile_buffer fb(fp);
    XDR_format<cfile_buffer> xffb(fb);
    obj_input<XDR_format<cfile_buffer> >
        stream(xffb);
\end{verbatim}}
    A stream can then be used to externalize, or internalize, any
    data-structure, by invoking the appropriate methods.

\subsection{Numbers}

All native C++ data types (numbers and booleans) are externalized and
internalized with method \texttt{simple}. For instance, an integer
declared as:
{\small\begin{verbatim}
    int i;
\end{verbatim}}
\noindent can be externalized or internalized by:
{\small\begin{verbatim}
    stream.simple(i);
\end{verbatim}}
    The kind of stream, \texttt{obj\_output} or \texttt{obj\_input},
    determines whether the operation performed is externalization or
    internalization.

\subsection{Composites}

To externalize composite data-structures (i.e. structures and classes)
a \emph{filter} must be defined. A filter is just a public template
method of the structure or class and is used both for externalization
and internalization. The body o the filter method enumerates and
describes each of the members of the composite.  As each method of a
stream returns a reference to itself, invocations on the same stream
can be cascaded, as in:
{\small\begin{verbatim}
    struct S {
        int a, b;
        template <class Stream>
        void composite(Stream& stream) {
            stream.simple(a).simple(b);
        }
    };
    class C {
        float a, b;
     public:
        template <class Stream>
        void composite(Stream& stream) {
            stream.simple(a).simple(b);
        }
    };
\end{verbatim}}
    
    After defining appropriate filters, instances of classes and
    structures can be externalized and internalized as native data
    types with method \texttt{simple}:

{\small\begin{verbatim}
    S s; C c;
    stream.simple(s).simple(c);
\end{verbatim}}
Furthermore, inheritance is supported by recursively calling the
externalization method on all base classes:
{\small\begin{verbatim}
    class D: public C {
        int d, e;
     public:
        template <class Stream>
        void composite(Stream& stream) {
            C::composite(stream);
            stream.simple(d).simple(e);
        }
    };
\end{verbatim}}
    However, this does not work if externalization is called through a
    pointer to the base class. In this situation, the runtime type of
    the object needs to be used, as described in Section
    \ref{sec:apiobjects}.

\subsection{Strings}

There are several representations of strings in C++, such as C-style
zero-terminated strings and container-style strings.  Zero-terminated
C-style strings are externalized with method \texttt{cstring}. If
space for the string is statically allocated, the \xtl\ must be
informed of the maximum size:
{\small\begin{verbatim}
    char s[LEN];
    stream.cstring(s, LEN);
\end{verbatim}}
 
    If the string is dynamic, the \xtl\ will always reserve the
    necessary space for the string. The previous value will be deleted
    with \verb|delete []| if not \texttt{NULL}, so is up to the
    programmer to ensure that it is properly initialized prior to
    internalization.
{\small\begin{verbatim}
    char* d=new char[LEN];
    stream.cstring(d);
\end{verbatim}}

Container-style C++ strings are internalized and externalized with
\texttt{simple}, just as any other composite:
{\small\begin{verbatim}
    string s;
    stream.simple(s);
\end{verbatim}}
    The external representation of all kinds of string is the same,
    and as such, interchangeable (e.g. a string can be externalized
    from a C string and internalized to a C++ string).

\subsection{Arrays}
\label{sec:arrays}

The \xtl\ supports two distinct kinds of arrays: statically allocated
with fixed size or dynamically allocated with variable size. Fixed
size arrays can be externalized and internalized with \texttt{vector}:
{\small\begin{verbatim}
    int a[10];
    stream.vector(a, 10);
\end{verbatim}}
    
    Variable sized arrays, composed of a scalar size and the array
    itself, are externalized and internalized with \texttt{array}:
{\small\begin{verbatim}
    int size=LEN;
    int* a=new int[size];
    stream.array(a, size);
\end{verbatim}}
    As happens with dynamic C-style strings, the \xtl\ will always
    reserve the necessary space for the array. The previous value will
    be deleted with \verb|delete []| if not \texttt{NULL}, so is up to
    the programmer to ensure that it is properly initialized prior to
    internalization.
    
    The \xtl\ supports only arrays of elements which can be
    externalized and internalized with \texttt{simple}.  If that is
    not the case, data must be wrapped within a structure with an
    appropriate filter.

Arrays of bytes which are already in a suitable external
representation (e.g. an well known graphics format), don't need to be
converted. In this situation, \texttt{vector} and \texttt{array}
should be replaced with \texttt{opaque} and \texttt{bytes},
respectively.

\subsection{Pointers}

Pointers in C++ can express references to data items which are always
present or optional data items, depending on the pointer being or not
allowed to be null.

References to data items which are always present can be internalized
and externalized as:
{\small\begin{verbatim}
    int* a=new int(10);
    stream.reference(a);
\end{verbatim}}
Optional data items can be internalized and externalized as:
{\small\begin{verbatim}
    int* a=(int*)0, *b=new int(10);
    stream.pointer(a).pointer(b);
\end{verbatim}}
    As happens with dynamic arrays and C-style strings, the \xtl\ will
    always reserve the necessary space for the pointed structure. The
    previous value will be deleted with \verb|delete| if not
    \texttt{NULL}, so is up to the programmer to ensure that it is
    properly initialized prior to internalization.  As with dynamic
    arrays, the referenced or optional data item must be
    externalizable and internalizable with \texttt{simple}.

\subsection{Templates}

Template types are supported by the \xtl\ without any further
complexity:
{\small\begin{verbatim}
    template <class T>
    class C {
        T a, b;
     public:
        template <class Stream>
        void composite(Stream& stream) {
            stream.simple(a).simple(b);
        }
    };
\end{verbatim}}
\noindent and are used just as any other composite data-type:
{\small\begin{verbatim}
    C<int> ci;
    C<float> cf;

    stream.simple(ci).simple(cf);
\end{verbatim}}
Notice that if \texttt{C} is instantiated with a type that is not
externalizable with \texttt{simple}, it will result in a compilation
error.

\subsection{Containers}
\label{sec:apicontain}

A special case of composites are the STL containers which do not have
the required filter methods. However, as they can be accessed through
their external interface it is possible to
externalize them with:
{\small\begin{verbatim}
    list<int> li;
    stream.container(li);
\end{verbatim}}  
    This filter may also be used for any other standard or even
    user-defined containers which support a forward iterator. The
    elements of the container must be externalizable with
    \texttt{simple}. In order to externalize \texttt{maps} you have to
    define an appropriate overload of global function
    \texttt{composite} for each instanciation of maps.
    
    If supported by the compiler\footnote{As of release 1.3, G++ 2.95
      only.} all containers, including maps, can be externalized with
    \texttt{simple}, without the need for defining filters for
    specific pairs.

\subsection{Unions}

Unions in \xtl\ must always be discriminated by an integer value to
select the appropriate externalization procedure:
{\small\begin{verbatim}
    int discr;
    union {
        int i;
        float f;
    } val;
\end{verbatim}}
\noindent Externalization is accomplished by calling \texttt{choices}:
{\small\begin{verbatim}
    stream.choices(discr, val.i, val.f);
\end{verbatim}}
\noindent    where the first element is the discriminator and the following
variable number of elements are the alternative values of the union.
The alternatives are numbered from 0 for the leftmost and counting up,
that is, the preceding example, 0 indicates an integer value and 1 a
floating point value.
    
Even if the use of unions is not very common in C++, they are useful
in the context of externalization to set a version number on the
externalized data and then to recognize multiple versions upon
internalization.

\subsection{Objects}
\label{sec:apiobjects}

Pointers to base classes are in the \xtl\ considered an implicitly
discriminated union of pointers to all possible derived classes of the
pointer type.  For instance, in the following example:
{\small\begin{verbatim}
    class B {
     public:
        virtual void f()=0;
    };
    class D1: public B {
        int a;
     public:
        virtual void f() {
            // something with a
        }
        template <class Stream>
        void composite(Stream& stream) {
            stream.simple(a);
        }
    };
    class D2: public B {
        float b;
     public:
        virtual void f() {
            // something with b
        }
        template <class Stream>
        void composite(Stream& stream) {
            stream.simple(b);
        }
    };
\end{verbatim}}
\noindent instances of both \texttt{D1} and \texttt{D2} can be externalized
  through a pointer to \texttt{B} with:
{\small\begin{verbatim}
    B* p=new D1;
    stream.object(p, (D1*)0, (D2*)0);
\end{verbatim}}
    Notice that the first parameter is the pointer itself and the
    following are all possible concretizations of \texttt{B}. As a
    consequence, if \texttt{B} was not abstract, it would have to be
    listed as a possible concretization of itself:
{\small\begin{verbatim}
    stream.object(p, (B*)0, (D1*)0, (D2*)0);
\end{verbatim}}
All parameters but the first are used just to instantiate the
template, so any value is correct, as long as it is of the correct
type.

\section{Advanced features}
\label{sec:adv}

\subsection{Third-party classes}
\label{sec:advspec}

When the externalization of some data-structure with \texttt{simple}
is required but it is not possible to add the filter method, it is
possible to achieve the same result by overloading
the global template function \texttt{composite}.

For example, to allow the externalization of STL lists with
\texttt{simple} all it takes is:
{\small\begin{verbatim}
    template <class Stream, class T>
    void composite(Stream& stream,
                   list<T>& list) {
        stream.container(list);
    }
\end{verbatim}}
    Overloading of this global function can also be used
    to override default filters defined as methods.

\subsection{Separating input and output}
\label{sec:sepio}

Sometimes it is not desired, or even possible, to use the same filter
for both externalization and internalization. For instance, with
structures which have a peculiar internal representation that does not
map directly to the desired external format.

One example is the externalization of the \xtl\ \texttt{mem\_buffer},
which is internally represented by three pointers into a buffer:
\begin{itemize}
\item \texttt{buffer} points to the start of the data;
\item \texttt{ptr} points to the next position to be read or written;
\item \texttt{lim} points to the first position past the end of the
  buffer.
\end{itemize}
The adequate external representation is obviously the same as the one
produced by \texttt{bytes}\footnote{See Section \ref{sec:arrays}.}
plus an index. However, it is impossible to use that filter because
the size is implicit in \texttt{lim-buffer}. To circumvent the
problem, when externalizing, a temporary variable must be used to
compute the size.  When internalizing, the size is used to compute
\texttt{lim}.

This can be done by defining filter methods separately for input and
output streams:

{\small\begin{verbatim}
    template <class Format>
    void composite(obj_input<Format>&
                             stream) {
        int size, idx;
        stream.bytes(buffer, size)
            .simple(idx);
        ptr=buffer+idx;
        lim=buffer+size;
    }
    template <class Format>
    void composite(obj_output<Format>&
                               stream) {
        stream.bytes(buffer, lim-buffer)
            .simple(ptr-buffer);
    }
\end{verbatim}}

\subsection{Pointer aliasing}
\label{cyclic}

Pointer aliasing is quite common in C++ data-structures. If this
possibility is not anticipated by the externalization code, multiple
copies of objects will be written to the external representation and
internalized as duplicates. Besides being a waste of space, it can
also be an error if the program depends on the aliasing for sharing,
which is often the case.

Even worse, if pointer aliasing results in a cyclic data-structure
(e.g. a circular linked list), an attempt to externalize it without
taking aliasing into consideration will result in an endless loop.

To circumvent this problem, the \xtl\ object stream can be
parameterized with a strategy class \cite{patterns}
\texttt{graph\_refs} which intercepts operations with pointers,
searching for duplicates upon externalization and correctly aliasing
pointers upon internalization:
{\small\begin{verbatim}
    obj_output<text_format<mem_buffer>,
                        graph_refs> os(mb);
\end{verbatim}}

\subsection{Smart-pointers}

Reference counting with smart-pointers is a common strategy for
managing pointer aliasing in C++. The \xtl\ supports the
externalization of these objects by simultaneously using the
techniques described in the previous two sections.

For example, typical externalization methods to be defined as part of
a smart-pointer to count references to objects of class \texttt{C}
would be:
{\small\begin{verbatim}
    template <class Format>
    void composite(obj_input<Format>&
                             stream) {
        if (ptr && !ptr->decref())
            delete ptr;
        ptr=(C*)0;
        stream.pointer(ptr);
        if (ptr) ptr->incref();
    }
    template <class Format>
    void composite(obj_output<Format>&
                              stream) {
        stream.pointer(ptr);
    }     
\end{verbatim}}
    The extra work of the internalization procedure ensures that the
    reference counts are correctly updated and avoids erroneous
    deletions of shared objects.

\subsection{Polymorphic streams}

Although the \xtl\ defaults to statically composed and optimized
streams which result in fast and fairly compact code, in certain
situations, it is desirable to take a different approach.

For instance, if many different combinations of buffer and format
drivers are used within the same program, the generated code grows
accordingly. In these situation, polymorphic wrappers for buffer
drivers reduce the size of the generated code to be proportional to
the number of distinct components and not the number of distinct
combinations. On the other hand, this wrapper reduces the possibility
of compiler optimizations and as such has some impact on performance.

\subsection{Generalized objects}
\label{sec:apiobjectspp}

Pointers to base classes as considered in Section \ref{sec:apiobjects} are
only adequate for small inheritance hierarchies. There is also the
possibility of using bigger hierarchies, although there is a small
performance penalty. To use this mechanism, all classes have to be
annotated with \texttt{decl\_externalizable(B)}. For instance, using
the same example class hierarchy:
{\small\begin{verbatim}
    class B {
        decl_externalizable(B);
     public:
        virtual void f()=0;
    };

    class D1: public B {
        decl_externalizable(D1);
        int a;
     public:
        virtual void f() {
            // something with a
        }
        template <class Stream>
        void composite(Stream& stream) {
            stream.simple(a);
        }
    };
    class D2: public B {
        decl_externalizable(D2);
        float b;
     public:
        virtual void f() {
            // something with b
        }
        template <class Stream>
        void composite(Stream& stream) {
            stream.simple(b);
        }
    };
\end{verbatim}}
\noindent In addition, in an implementation file you need to create an
index of classes that can be externalized:
{\small\begin{verbatim}
externalizable_index idx;
impl_externalizable(D1, 1, idx);
impl_externalizable(D2, 2, idx);
\end{verbatim}}
\noindent Instances of both \texttt{D1} and \texttt{D2} can be externalized
  through a pointer to \texttt{B} with:
{\small\begin{verbatim}
    B* p=new D1;
    stream.auto_object(p);
\end{verbatim}}

\section{Performance}
\label{sec:perf}

As the \xtl\ does not rely on a special compiler to generate efficient
externalization procedures, some care must be taken to ensure that the
code written by application programmers is efficient.

This goal is achieved by making sure that, unless the programmer
explicitly requires otherwise, all variables and function calls are
bound at compile time. As such, even though the \xtl\ code is deeply
nested, all functions can be inlined, which means that deep
invocations can be collapsed to sequential code.

As a consequence, the compiler has the opportunity to perform
extensive optimization, both for speed as well as for code size, by
reordering instructions and removing unused code.

The result is that externalization code written with the \xtl\ is very
fast, when compared to a similar library where these optimizations are
not possible. Table \ref{tab:perf} shows measurements of the time
necessary to externalize a deeply nested complex data-structure using
different format drivers. The tests were run on a Pentium Pro 200MHz
machine running Linux 2.0 and using the \texttt{egcs} 1.0.2 compiler
both with and without optimization. The size of the external
representation of the data-structure used for testing is about 400
bytes, depending on the format used.

\begin{table}[t]
  \begin{center}
    \begin{tabular}{rcc}
\hline\hline
                    &  optimized & not optimized \\
\hline
             memcpy & 4.0 & 5.0 \\
            SunXDR & 35.0 & 36.6 \\
     \hline
no conversion & 10.1 & 63.7 \\
   XDR & 14.2 & 73.7 \\
  GIOP CDR& 12.8 & 89.2 \\
\hline
    \end{tabular}
    \caption{Performance measurements. All times in micro-seconds.}
    \label{tab:perf}
  \end{center}
\end{table}

As it can be seen, using the \xtl\ with no format conversion results
in times very close to the theoretical maximum, measured by copying the
same amount of data with \texttt{memcpy}.

It is also notable that when converting to XDR format, the optimized
\xtl\ code is more than twice as fast as the code produced by
\texttt{rpcgen} for the same structure. Notice that the \xtl\ version
required absolutely no effort by the programmer to optimize it while
the code produced by \texttt{rpcgen} already uses some explicit inline
macros, which makes it more complex.

Finally, if we compare the effect of compiler optimization, it can be
seen that the code produced by \texttt{rpcgen} is only marginally
affected by optimization, which is the consequence of dynamic binding,
while with \xtl\ the effect is as much as six fold.

These facts confirm that C++ is adequate for performance sensitive
system-level operations, if the developer has the knowledge to select
which of the features of the language are appropriate.

\section{Applications}
\label{sec:extre}

Most of the experience with the \xtl\ is based on a previous version
of the library. Although it supported only text format and had the
lower two layers written in C, the upper layer was already written in
C++ and closely followed the current \xtl, making them differ mostly
on performance.

This library has been used in several projects, both for persistence
and communication. For persistence, it has been used to store
configuration and data in an object replication system targeted for
mobile hosts and to hold the cache for a directory service proxy for
disconnected operation \cite{brep}. In both these applications, the
flexibility and power of the \xtl\ interface have clearly offset the
disadvantages of not using a data-description language compiler.

For communication, it has been used as the basis for a remote
procedure call library. The use of a human readable external data
format has proved to be very useful when debugging, as it was possible
to interactively establish a connection to a server, manually issue
procedure invocations and immediately examine the results.  The \xtl\ 
makes it possible do the same while developing and then switch to a
more compact data format without compromising on efficiency.

The \xtl\ is being used in the development of a reliable multicast
protocol.  Besides being directly used in the conversion of messages
to its external representation, the concepts that were applied to the
\xtl\ are also being used to allow that highly configurable protocols
are easily assembled by application programmers without incurring in
unacceptable overheads.

The \xtl\ is also being used in the LyX
project\footnote{\texttt{http://www.lyx.org}} to handle communications
between the graphical user interface and the core.

Other application being considered for the \xtl\ is to provide
light-weight and high-performance interoperability with CORBA systems
at low-level, both with GIOP based protocols, such as the Internet
Inter-ORB Protocol, as well as with the Externalization Service.

The \xtl\ could also be used as an efficient multi-format back-end to
simplify the development of a data description compiler, which would
not require optimization of externalization procedures and would
easily work with hand-written code when necessary.

It is also possible to use custom streams, format or buffer drivers
which use the traversal procedures for purposes other than
externalization. An example is the description format driver, which
outputs an abstract description of the structure as text. This feature
allowed a remote procedure server to describe itself to a client,
which would then generate stubs on-the-fly. Other examples are buffer
and format drivers that skip some input or calculate the size of a
data-structure in a specific format, making it possible to perform
random access to externalized data. Another possible application, but
that has not been implemented, would be to use the traversal to mark
reachable structure for garbage collection.

These applications show that externalization procedures written with
the \xtl\ are in fact generic traversals of data-structures and as
such can be used as an implementation of the visitor pattern
\cite{patterns} for purposes other than externalization.

\bibliographystyle{plain}
\bibliography{xtl}

\end{document}


